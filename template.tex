\documentclass[12pt]{article}
\usepackage[utf8]{inputenc}
\usepackage{hyperref}
\hypersetup{
	linktoc=all
}

\title{TAkSiGaYs Kannatusjäsenten Gokous [DATE]} % ISO 8601 format preferred
\author{}

\begin{document}
\maketitle
 
\tableofcontents

\section{Kokouksen avaus}

\begin{itemize}
	\item{\textbf{Paikka} [PLACE]} % Where was the meeting held?
	\item{\textbf{Aika} [DATE]} % Starting time of the meeting. Example: 2020-01-18T21:50
	\item{\textbf{Puheenjohtaja} [CHAIR]} % Name of the chair of the meeting
	\item{\textbf{Sihteeri} [SECRETARY]} % Name of the secretary
\end{itemize}

\section{Kokouksen laillisuus ja päätösvaltaisuus}

% Below usually applies, change if needed
Kokous on kutsuttu koolle vähintään minuuttia ennen kokousta.
Paikalla on vähintään yksi kannatusjäsen.

Kokous todetaan lailliseksi ja päätäntävaltaiseksi.

\section{Ilmoitusasiat ja posti}
\begin{itemize}
	% Add announcements here
	\item{}
\end{itemize}

\section{Uusien kannatusjäsenten hyväksyminen}

% Uncomment and remove itemization if no new members
% Ei uusia kannatusjäseniä

\begin{itemize}
	\item{}
\end{itemize}

\section{Muut esille tulevat asiat}

\begin{itemize}
	\item{}

\end{itemize}

\section{Kokouksen päättäminen}
Kokous päätettiin ajassa [END TIME]. % When was the meeting ended?

\end{document}
